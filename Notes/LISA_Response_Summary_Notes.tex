\documentclass[10pt,a4paper]{article}
\usepackage[utf8]{inputenc}
\usepackage{amsmath}
\usepackage{amsfonts}
\usepackage{amssymb}
\usepackage{graphicx}
\usepackage[spaces,hyphens]{url}
\usepackage[colorlinks,allcolors=blue]{hyperref} 
\title{LISA Response summary}
\begin{document}
\maketitle


From section 3.1 of \cite{cornish}, we get the varying portion for the round trip to be,

\begin{equation}
\delta l(t_2) = \frac{L sin^2 \theta}{2} \ \tau(cos \theta,f) \left[ h_{+}(t_2) cos 2\psi + h_{x}(t_2) sin 2\psi \right]
\end{equation}
Where, we have considered the arm to be in the x-z plane $ \textbf{u} = \hat{x} \ sin \theta + \hat{z} \ cos \theta$ \\


$\psi$ is the polarisation angle. $\tau$ is the transfer function as defined in eq(7) of \cite{cornish}

Now from we consider the plus and cross polarisation amplitudes of a binary black hole with an angle of inclination $i$ with the line of sight $\hat{z}$ (3.27a and 3.27b) of \cite{whelan}

We can then write,

\begin{equation}
h(t) = \frac{A(t)}{d}\ \left[ F_{+} \ \frac{1+ cos^2 i}{2} cos\Phi(t) + F_{\times} \ cosi \ sin\Phi(t) \right]
\end{equation}

Where the $F_{+} and F_{x}$ are the antenna response functions

In the present case we can write 

\begin{align}
\delta l&=\frac{A(t) L sin^2 \theta \tau(cos \theta,f)}{2d} \left[\frac{1+cos^2 i}{2} \ cos 2\psi \ cos \Phi(t) + cos i \ sin 2\psi \ sin \Phi(t)\right]\\
&=\frac{A(t) L sin^2 \theta \tau(cos \theta,f)}{2d} \left( \sqrt{ \frac{(1+cos^2 i)^2}{4} cos^2 2\psi  + cos^2 i sin^2 2\psi} \right)cos(\Phi(t)-\Psi)
\end{align}


The general formula is given by eq(8) of \cite{cornish}

FOr a general $\hat{\Omega}$ specified by $\theta,\omega$ in spherical coordinate system, we find that the antenna pattern becomes:

\begin{align}
F_{+}(\theta,\phi,\psi,i)&=\frac{(1+cos^2 i)}{2} \left[\frac{1}{2}(1+ cos^2 \theta) cos2\phi cos\psi - cos\theta sin2\phi sin2\psi \right]\\
F_{\times}(\theta,\phi,\psi,i)&=cos i \left[\frac{1}{2}(1+ cos^2 \theta) cos2\phi cos\psi + cos\theta sin2\phi cos2\psi \right]\\ 
\end{align}

If we put back $\theta=0$ and $\phi=0$ in [5] and [6] we get back the responses in eq [4].

The measured strain will be (as from eq 14 and 15)

\begin{align}
s&=\frac{\delta l_u (t) - \delta l_v (t)}{l}\\
&=\textbf{D}(\hat{\Omega},f) : h(\hat{\Omega},f,x,t)
\end{align}

where $D(\hat{\Omega},f) = \frac{1}{2}\left((u \otimes u) \tau - (v \otimes v) \tau \right)$

Depending on the orientation of the two arms specified by the u and v vector we get the final response as a function of the arm orientation,the orientation and location of binary.
\begin{thebibliography}{9}

\bibitem{cornish}				\url{https://arxiv.org/pdf/gr-qc/0103075.pdf}

\bibitem{whelan} \url{https://dcc.ligo.org/public/0106/T1300666/003/Whelan_notes.pdf}
\end{thebibliography}

\end{document}