\documentclass[10pt,a4paper]{article}
\usepackage[utf8]{inputenc}
\usepackage{amsmath}
\usepackage{amsfonts}
\usepackage{amssymb}
\usepackage{graphicx}
\usepackage{float}
\usepackage{breqn}
\usepackage{natbib}
\usepackage[spaces,hyphens]{url}
\usepackage[colorlinks,allcolors=blue]{hyperref} 

\def\sinc{\mathop{\rm sinc}\nolimits}

\title{Project Notes}
\begin{document}
\maketitle

\section*{GW signal from a binary}

A binary emits GW as follows.
\begin{equation}
\begin{pmatrix}
  h_+\\h_\times
\end{pmatrix}
= \frac{4G\mu v^2}{c^4 D}
\begin{pmatrix}
  \frac12(1+\cos^2\iota) \\
  \cos\iota
\end{pmatrix}
\exp(i\omega t)
\end{equation}
See Figure~\ref{fig:binary}

\begin{figure}[h]
\centering
\includegraphics[scale=0.25]{../Figures/diagram1.jpg}
\caption{The one arm arrangement with the source in the z-axis with
  angle of inclination $\iota$ and the arm along the vector \textbf{u}
  which lies in the x-z plane.\label{fig:binary}}
\end{figure}

LISA's response function can be calculated by imagining it as a two arm Michelson interferometer. We first derive the response of a single arm and then cross correlate with the other to measure the total strain.

Consider two LISA spacecraft, one ar the origin and another at a distance L away from it.  We choose the coordinate axes such that the spacecrafts lie in the x-z plane. The vector  $ \textbf{u} = \hat{x} \ \sin \theta + \hat{z} \ \cos \theta$ describes the single arm here.

The spacecraft at origin sends a series of photons while the weak plane gravitational wave passes through space in the +z direction. Later we will generalize the source to be in a random direction.

We can write the space-time metric in the transverse-traceless gauge
(TT-Gauge). Here $\psi$ refers to the polarization angle.

\begin{equation}
\begin{aligned}
ds^2 &= -c^2 dt^2 + dz^2 \\
&+ (1 + h_+\cos2\psi + h_\times\sin2\psi) \, dx^2 \\
&+ (1 - h_+\cos2\psi - h_\times\sin2\psi) \, dy^2 \\
&- 2(h_+\sin2\psi - h_\times\cos2\psi) \, dx\,dy
\end{aligned}
\end{equation}

Using this metric we can  find the photon path similar to
\cite{cornish}.
Using that we compute the round trip journey from spacecraft 1 to spacecraft 2 and back again.

\begin{equation}
  \frac{\delta L}L = \tau(\omega,\theta,L) \,
  \Big( h_+ \, \cos2\psi + h_\times \sin2\psi \Big)
\end{equation}
where
\begin{equation}
\begin{aligned}
  \tau(\omega,\theta,L) \equiv {\textstyle\frac12} \sin^2\theta
  \Big( &\sinc(\alpha(1-\cos\theta))\,\exp(-i\alpha(3+\cos\theta)) \, + \\
        &\sinc(\alpha(1+\cos\theta))\,\exp(-i\alpha(1+\cos\theta)) \Big)
\end{aligned}  
\end{equation}
and $\alpha\equiv(\omega L)/(2c)$.

In the low frequency limit, the transfer function $\tau$ is just of the order of one.
The total strain therefore depends on the angle of orientation of the
arm $\theta$, the angle of inclination of the orbit $\iota$, the
location of the source(in this case we took it as z-axis) and the
polarization angle $\psi$.  See Figure \ref{fig:resporient}.

A general coordinate free formula is given by eq(8) in \cite{cornish}.\\
 
In general the strain $h(t)$ can be written as a linear combination of $h_{+}(t)$ and $h_{\times}(t)$.

\begin{equation}
h(t) = F_{+}(\theta_s,\phi_s,\psi_s) h_{+}(t) + F_{\times}(\theta_s,\phi_s,\psi_s)
\end{equation}

We can rewrite the signal from the binary in the conventional amplitude-and-phase form.

\begin{equation}
h(t) = -A(t) \cos[2\Phi(t) +\varphi]
\end{equation}

where $A(t)$ is given by



\begin{equation}
A(t) = \frac{2 G \mu M}{c^4 a D}\left([1+(\hat{\textbf{L}} \cdot \hat{\textbf{N}})^2]^2 F_{+}^2(\theta_s,\phi_s,\psi_s) + 4[\hat{\textbf{L}} \cdot \hat{\textbf{N}}]^2 F_{\times}^2(\theta_s,\phi_s,\psi_s) \right)^{\frac{1}{2}}
\end{equation}

Where $\hat{\textbf{L}}$ and $\hat{\textbf{N}}$ are the unit vectors along the angular momentum of the binary and the one along the line of sight in the detector frame respectively. $\phi(t)$ represents the phase. 
Also, $\hat{\textbf{L}} \cdot \hat{\textbf{N}} = \cos \ \iota$ where $\iota$ is the inclination of the binary. These are defined in the Figure \ref{fig:directions} from \citep{ACST}.\\ 

The "detector beam patterns" $F_{+}$ and $F_{\times}$ depend on the source direction and the  polarization angle $\psi$.\citep{cutler}. The polarization angle $\psi$ is the angle from the principal + direction ($\hat{\textbf{N}} \times \hat{\textbf{L}}$) , clockwise in the plane of sky to the azimuth($\hat{\textbf{N}} \times (\hat{\textbf{N}} \times \hat{\textbf{z}})$)

\begin{align}
F_{+}(\theta_s,\phi_s,\psi_s)&=\frac{1}{2}(1+ \cos^2\theta_s) \ \cos2\phi_s \ \cos2\psi_s - \cos\theta_s \ \sin 2\phi_s \ \sin 2\psi_s\\
F_{\times}(\theta_s,\phi_s,\psi_s)&=\frac{1}{2}(1+ \cos^2\theta_s) \ \cos2\phi_s \ \cos2\psi_s + \cos\theta_s \ \sin 2\phi_s \ \sin 2\psi_s
\end{align}


\begin{figure}[!h]
\centering
\includegraphics[scale=0.6]{../Figures/directions.pdf}
\caption{The definitions of various directions used in the equations. Left figure shows how the binary projected on the sky looks like.\label{fig:directions}}
\end{figure}

\begin{figure}[!h]
\centering
\includegraphics[scale=0.3]{../Figures/responsevsorientationofbinary.pdf}
\caption{The response $\delta l/L$ varying with the polarization
  angle $\psi$ for various inclination angle.\label{fig:resporient}}
\end{figure}

\newpage
\section*{LISA Verification Binaries}
The Laser Interferometer Space Antenna(LISA) will be the first gravitational wave observatory in space. LISA will be operating in the low frequency part of the gravitational wave spectrum($10^4$--1 Hz). In this range, we expect to observe lots of ultracompact binaries with orbital periods shorter than few hours. Out of these UCBs, AM CVn type binaries are of particular interest. Due to therir strong GW signals, they are guaranteed to be  detetcted on LISA band. These are termed 'verification binaries'.


\begin{figure}[ht]
\centering
\includegraphics[scale=0.25]{../Figures/strain_verific_binary.pdf}
\caption{The calculated strains due to LISA verification binaries and the sensitivity curve of LISA}
\end{figure}
\section*{HP Lib Verfication Binary}

We are interested in providing an independent prediction for $i$ and $\Psi$ for the AM CVn binary HP Lib. The binary consist of of on high mass white dwarf and a low mass star(brown dwarf).\\

\begin{table}[H]
\centering
\begin{tabular}{|c|c|c|c|}
\hline 
\rule[-1ex]{0pt}{2.5ex} Source & m1($M_{\odot}$) & m2($M_{\odot}$) & $P_{orb}$(sec) \\ 
\hline 
\rule[-1ex]{0pt}{2.5ex} HP Lib & 0.49-0.80 & 0.048-0.08 & 1102.70 \\ 
\hline 
\end{tabular}
\caption{The estimated values of mass and time period of HP Lib}
\end{table}

\begin{figure}[ht]
\centering
\includegraphics[scale=0.5]{diagram1.png}
\label{1}
\caption{The binary system HP Lib}
\end{figure}

Consider the verification binary HP Lib with masses $m_1$ and $m_2$ and period P \ref{1}. We can find the fraction of light recieved by the brown dwarf $m_2$ in the following way:

The flux from mass $m1$ obeys the inverse square law. Assuming a luminosity L, the flux at a distance d is given by $L/4 \pi d^2$. The cross section area for brown dwarf is $\pi R^2$.


Therefore, the total flux recieved is: $$\frac{L}{4 \pi d^2} \ \pi R^2 = \frac{L}{4} \ \left(\frac{R}{d}\right)^2 $$ where R is the radius of the brown dwarf and d is the seperation.

We can estimate d from the time period of the orbit. We know from Kepler's 3rd law that, $T^2 = \frac{4 \pi^{2}}{G (m_1 + m_2)} d^3$. We will denote $M=m_1 + m_2$

\begin{align*}
\frac{d}{c}&=\left(\frac{GM}{c^3}\right)^{1/3} \frac{T^{2/3}}{(4 \pi^{2})^{1/3}}\\
&=\frac{\left(\frac{GM}{c^3}\right)^{1/3}}{\left(\frac{GM_{\odot}}{c^3}\right)^{1/3}} \frac{T^{2/3}}{(4\pi^2)^{1/3}} (5 \times 10^{-6})^{1/3}
\end{align*}

Now the fraction of the light recieved is,

\begin{equation}
\frac{1}{4} \left(\frac{R}{c}\right)^2 \left(\frac{M}{M_{\odot}}\right)^{-2/3} \frac{T^{-4/3}}{(4 \pi^2)^{-2/3}} (5 \times 10^{-6})^{-2/3}
\end{equation}

For the first estimate, we can take $R/c \approx 0.1$ which if we subtitute above give an estimate of the light received to be 0.01315

\bibliographystyle{aa}
\bibliography{refs.bib}

\end{document}

\begin{thebibliography}{9}

\bibitem{cornish}				\url{https://arxiv.org/pdf/gr-qc/0103075.pdf}

\bibitem{whelan} \url{https://dcc.ligo.org/public/0106/T1300666/003/Whelan_notes.pdf}
\end{thebibliography}






\begin{multline}
ds^2 = - c^2 dt^2 + (1+ h_+ \cos 2\psi + h_x \sin 2 \psi) d x^2 +(1- h_+ \cos 2\psi - h_x \sin 2 \psi) d y^2 \\
 - 2(h_+ \sin 2 \psi - h_x \cos 2 \psi) dx dy + dz^2
\end{multline}

\begin{equation}
\label{resp}
\delta l(t_2) = \frac{L \sin^2 \theta}{2} \ \tau(\cos \theta,f) \left[ h_{+}(t_2) \cos 2\psi + h_{x}(t_2) \sin 2\psi \right]
\end{equation}

$\tau$ is the transfer function as defined in eq(7) of \cite{cornish}

Now from we consider the plus and cross polarisation amplitudes of a binary black hole with an angle of inclination $i$ with the line of sight $\hat{z}$ (3.27a and 3.27b) \cite{whelan}. 

Substituting the $h_{+}$ and $h_{x}$ expressions, eq [\ref{resp}] becomes,
\begin{align}
\delta l&=\frac{A(t) L \sin^2 \theta \tau(\cos \theta,f)}{2d} \left[\frac{1+\cos^2 i}{2} \ \cos 2\psi \ \cos \Phi(t) + \cos i \ \sin 2\psi \ \sin \Phi(t)\right]\\
&=\frac{A(t) L \sin^2 \theta \tau(\cos \theta,f)}{2d} \left( \sqrt{ \frac{(1+\cos^2 i)^2}{4} \cos^2 2\psi  + \cos^2 i \sin^2 2\psi} \right)\cos(\Phi(t)-\Psi)
\end{align}

Where $\Phi(t) = 2 \pi f t + \Phi_0$.

For a general location of the binary ($\hat{\Omega}$) specified by $\theta_s,\omega$ in spherical coordinate system, we can derive the antenna beam patterns.\cite{whelan}

\begin{align*}
F_{+}(\theta_s,\phi,\psi,i)&=\frac{(1+\cos^2 i)}{2} \left[\frac{1}{2}(1+ \cos^2 \theta_s) \cos2\phi \cos\psi - \cos\theta_s \sin2\phi \sin2\psi \right]\\
F_{\times}(\theta_s,\phi,\psi,i)&=\cos i \left[\frac{1}{2}(1+ \cos^2 \theta_s) \cos2\phi \cos\psi + \cos\theta \sin2\phi \cos2\psi \right]\\ 
\end{align*}

If we put back $\theta_s=0$ and $\phi=0$ in these equations we get back the responses in eq [4].

We can form an interferometer by introducing a third spacecraft at a distance L from the corner spacecraft and subrating the outputs of the two arms \cite{cornish} to get the total strain.


\begin{align}
s(\hat{\Omega},f,\textbf{x},t)&=\frac{\delta l_u (t) - \delta l_v (t)}{l}\\
&=\textbf{D}(\hat{\Omega},f) : h(\hat{\Omega},f,\textbf{x},t)
\end{align}

where $$D(\hat{\Omega},f) = \frac{1}{2}\left((u \otimes u) \tau - (v \otimes v) \tau \right)$$

is the detector response tensor. \textbf{u} and \textbf{v} are the unit vectors representing the direction of each interferometer arm. $\hat{\Omega}$ is the direction of propegation of the wave. We can use the orientation of LISA arms and compute the detetctor response functions $F_{+}$ and $F_{\times}$.




